% Options for packages loaded elsewhere
\PassOptionsToPackage{unicode}{hyperref}
\PassOptionsToPackage{hyphens}{url}
%
\documentclass[
]{article}
\usepackage{amsmath,amssymb}
\usepackage{iftex}
\ifPDFTeX
  \usepackage[T1]{fontenc}
  \usepackage[utf8]{inputenc}
  \usepackage{textcomp} % provide euro and other symbols
\else % if luatex or xetex
  \usepackage{unicode-math} % this also loads fontspec
  \defaultfontfeatures{Scale=MatchLowercase}
  \defaultfontfeatures[\rmfamily]{Ligatures=TeX,Scale=1}
\fi
\usepackage{lmodern}
\ifPDFTeX\else
  % xetex/luatex font selection
\fi
% Use upquote if available, for straight quotes in verbatim environments
\IfFileExists{upquote.sty}{\usepackage{upquote}}{}
\IfFileExists{microtype.sty}{% use microtype if available
  \usepackage[]{microtype}
  \UseMicrotypeSet[protrusion]{basicmath} % disable protrusion for tt fonts
}{}
\makeatletter
\@ifundefined{KOMAClassName}{% if non-KOMA class
  \IfFileExists{parskip.sty}{%
    \usepackage{parskip}
  }{% else
    \setlength{\parindent}{0pt}
    \setlength{\parskip}{6pt plus 2pt minus 1pt}}
}{% if KOMA class
  \KOMAoptions{parskip=half}}
\makeatother
\usepackage{xcolor}
\usepackage[margin=1in]{geometry}
\usepackage{longtable,booktabs,array}
\usepackage{calc} % for calculating minipage widths
% Correct order of tables after \paragraph or \subparagraph
\usepackage{etoolbox}
\makeatletter
\patchcmd\longtable{\par}{\if@noskipsec\mbox{}\fi\par}{}{}
\makeatother
% Allow footnotes in longtable head/foot
\IfFileExists{footnotehyper.sty}{\usepackage{footnotehyper}}{\usepackage{footnote}}
\makesavenoteenv{longtable}
\usepackage{graphicx}
\makeatletter
\def\maxwidth{\ifdim\Gin@nat@width>\linewidth\linewidth\else\Gin@nat@width\fi}
\def\maxheight{\ifdim\Gin@nat@height>\textheight\textheight\else\Gin@nat@height\fi}
\makeatother
% Scale images if necessary, so that they will not overflow the page
% margins by default, and it is still possible to overwrite the defaults
% using explicit options in \includegraphics[width, height, ...]{}
\setkeys{Gin}{width=\maxwidth,height=\maxheight,keepaspectratio}
% Set default figure placement to htbp
\makeatletter
\def\fps@figure{htbp}
\makeatother
\setlength{\emergencystretch}{3em} % prevent overfull lines
\providecommand{\tightlist}{%
  \setlength{\itemsep}{0pt}\setlength{\parskip}{0pt}}
\setcounter{secnumdepth}{-\maxdimen} % remove section numbering
\usepackage{booktabs}
\usepackage{longtable}
\usepackage{array}
\usepackage{multirow}
\usepackage{wrapfig}
\usepackage{float}
\usepackage{colortbl}
\usepackage{pdflscape}
\usepackage{tabu}
\usepackage{threeparttable}
\usepackage{threeparttablex}
\usepackage[normalem]{ulem}
\usepackage{makecell}
\usepackage{xcolor}
\ifLuaTeX
  \usepackage{selnolig}  % disable illegal ligatures
\fi
\usepackage{bookmark}
\IfFileExists{xurl.sty}{\usepackage{xurl}}{} % add URL line breaks if available
\urlstyle{same}
\hypersetup{
  pdftitle={Data Analysis Report - Team 9},
  hidelinks,
  pdfcreator={LaTeX via pandoc}}

\title{Data Analysis Report - Team 9}
\author{}
\date{\vspace{-2.5em}}

\begin{document}
\maketitle

\subsection{1. Summary of Descriptive Statistics on Average
Ratings}\label{summary-of-descriptive-statistics-on-average-ratings}

In this section, we present a summary of the descriptive statistics for
the average ratings of TV shows. This analysis provides an overview of
key metrics such as the mean, median, standard deviation, and range of
ratings.

\begin{longtable}[]{@{}
  >{\raggedleft\arraybackslash}p{(\columnwidth - 8\tabcolsep) * \real{0.2899}}
  >{\raggedleft\arraybackslash}p{(\columnwidth - 8\tabcolsep) * \real{0.2029}}
  >{\raggedleft\arraybackslash}p{(\columnwidth - 8\tabcolsep) * \real{0.1594}}
  >{\raggedleft\arraybackslash}p{(\columnwidth - 8\tabcolsep) * \real{0.1594}}
  >{\raggedleft\arraybackslash}p{(\columnwidth - 8\tabcolsep) * \real{0.1884}}@{}}
\toprule\noalign{}
\begin{minipage}[b]{\linewidth}\raggedleft
mean\_average\_rating
\end{minipage} & \begin{minipage}[b]{\linewidth}\raggedleft
median\_rating
\end{minipage} & \begin{minipage}[b]{\linewidth}\raggedleft
min\_rating
\end{minipage} & \begin{minipage}[b]{\linewidth}\raggedleft
max\_rating
\end{minipage} & \begin{minipage}[b]{\linewidth}\raggedleft
num\_episodes
\end{minipage} \\
\midrule\noalign{}
\endhead
\bottomrule\noalign{}
\endlastfoot
7.395257 & 7.5 & 1 & 10 & 729601 \\
\end{longtable}

\section{2. Regressing Rating on Number of
Episode}\label{regressing-rating-on-number-of-episode}

For this analysis, we decided to use regression analysis to explore the
relationship between the number of episodes and the ratings of TV shows.
By applying this method, we aim to determine whether the number of
episodes significantly impacts a show's rating and to quantify the
strength of this relationship. This approach will help us gain insights
into how episode count influences audience engagement and reception.

\subsubsection{2.1 Main Model Without Control
Variables}\label{main-model-without-control-variables}

\begin{longtable}[t]{lrrrr}
\caption{\label{tab:unnamed-chunk-5}Model Summary: Number of Episodes on Average Rating}\\
\toprule
term & estimate & std.error & statistic & p.value\\
\midrule
(Intercept) & 7.3984538 & 0.0013401 & 5520.67262 & 0\\
Number\_of\_episodes & -0.0000678 & 0.0000027 & -24.71563 & 0\\
\bottomrule
\end{longtable}

\includegraphics{Data_Analysis_files/figure-latex/unnamed-chunk-5-1.pdf}

\subsubsection{2.2 Regression Analysis
Output}\label{regression-analysis-output}

\begin{itemize}
\tightlist
\item
  Coefficient for Number of episodes: -0.6819 x 10\^{}-5
\item
  T-value for Number of episodes: -24.62
\item
  P-value for Number of episodes: \textless2.2 x 10\^{}-16
\item
  R-squared: 0.0008274
\end{itemize}

In our basic model without any control variables, we can see that the
number of episodes have a slightly negative effect on the average IMDb
rating. With a P-value smaller than significance level of 5\%, we can
conclude that the number of episodes has a negative effect. However this
model is without any control variables, so we need to expand our model.

\subsubsection{2.3 Main Model With Control
Variables}\label{main-model-with-control-variables}

\begin{longtable}[t]{lrrrr}
\caption{\label{tab:unnamed-chunk-6}Model Summary: Number of Episodes on Average Rating}\\
\toprule
term & estimate & std.error & statistic & p.value\\
\midrule
(Intercept) & 7.4112214 & 0.0039325 & 1884.613686 & 0.0000000\\
Number\_of\_episodes & -0.0000074 & 0.0000033 & -2.272281 & 0.0230700\\
popularity & 0.6388453 & 0.0062147 & 102.796383 & 0.0000000\\
runtimeshort & -0.0111033 & 0.0037682 & -2.946624 & 0.0032128\\
new\_vs\_oldold & -0.0273492 & 0.0031471 & -8.690212 & 0.0000000\\
\addlinespace
episode\_quantityMany & -0.3178491 & 0.0047768 & -66.539811 & 0.0000000\\
\bottomrule
\end{longtable}

\paragraph{Control Variable
Definition:}\label{control-variable-definition}

\begin{itemize}
\tightlist
\item
  popularity: ``Amount of votes are over 1000
\item
  runtime: ``Runtime in minutes is more than 50''
\item
  new\_vs\_old: ``The start year is later than 2015''
\item
  episode\_quantity: ``Number of episodes is more than 25''
\end{itemize}

\subsubsection{2.4 Regression Analysis
Output:}\label{regression-analysis-output-1}

\begin{itemize}
\tightlist
\item
  Coefficient for Number of episodes: -1.183 x 10\^{}-6
\item
  T-value for Number of episodes: -0.355
\item
  P-value for Number of episodes: 0.722
\item
  R-squared 0.0376
\end{itemize}

In our main model with control variables, we observe that the
coefficient for the number of episodes is negative; however, it is not
significant, as the p-value for this variable is 0.772, which is greater
than 0.05. Looking at our control variables, we find that all of them
are significant: Popularity (amount of votes over 1000) has a
significant positive effect on the average rating. In contrast, runtime
(runtime in minutes is more than 50) has a significant negative effect
on the average rating. Additionally, being new (the start year is later
than 2015) has a significant negative effect on the average rating, and
having many episodes (more than 25 episodes) also negatively affects the
average rating.

\subsection{3. Correlation Matrix of the Predictive Variables in our
Main
Model}\label{correlation-matrix-of-the-predictive-variables-in-our-main-model}

\begin{longtable}[]{@{}
  >{\raggedright\arraybackslash}p{(\columnwidth - 10\tabcolsep) * \real{0.2121}}
  >{\raggedleft\arraybackslash}p{(\columnwidth - 10\tabcolsep) * \real{0.1919}}
  >{\raggedleft\arraybackslash}p{(\columnwidth - 10\tabcolsep) * \real{0.1111}}
  >{\raggedleft\arraybackslash}p{(\columnwidth - 10\tabcolsep) * \real{0.1313}}
  >{\raggedleft\arraybackslash}p{(\columnwidth - 10\tabcolsep) * \real{0.1414}}
  >{\raggedleft\arraybackslash}p{(\columnwidth - 10\tabcolsep) * \real{0.2121}}@{}}
\toprule\noalign{}
\begin{minipage}[b]{\linewidth}\raggedright
\end{minipage} & \begin{minipage}[b]{\linewidth}\raggedleft
Number\_of\_episodes
\end{minipage} & \begin{minipage}[b]{\linewidth}\raggedleft
popularity
\end{minipage} & \begin{minipage}[b]{\linewidth}\raggedleft
runtimeshort
\end{minipage} & \begin{minipage}[b]{\linewidth}\raggedleft
new\_vs\_oldold
\end{minipage} & \begin{minipage}[b]{\linewidth}\raggedleft
episode\_quantityMany
\end{minipage} \\
\midrule\noalign{}
\endhead
\bottomrule\noalign{}
\endlastfoot
Number\_of\_episodes & 1.0000000 & -0.0173176 & 0.0222762 & -0.0402571 &
0.2172219 \\
popularity & -0.0173176 & 1.0000000 & -0.0013969 & -0.0385738 &
-0.0608567 \\
runtimeshort & 0.0222762 & -0.0013969 & 1.0000000 & 0.0486213 &
-0.0101750 \\
new\_vs\_oldold & -0.0402571 & -0.0385738 & 0.0486213 & 1.0000000 &
0.0207109 \\
episode\_quantityMany & 0.2172219 & -0.0608567 & -0.0101750 & 0.0207109
& 1.0000000 \\
\end{longtable}

\subsubsection{Correlation Matrix
Analysis:}\label{correlation-matrix-analysis}

\begin{itemize}
\tightlist
\item
  Number of episodes has a moderate positive correlation (0.217) with
  episode quantityMany, indicating that shows with many episodes tend to
  be classified as having ``many'' episodes.
\item
  Number of episodes has weak correlations with other variables like
  popularity (-0.017), runtime (0.022), and new vs old (-0.040),
  suggesting that the number of episodes is not strongly related to
  these variables.
\item
  Popularity is weakly and negatively correlated with both new vs old
  (-0.039) and episode quantityMany (-0.061), indicating that neither
  older shows nor those with many episodes are strongly related to
  popularity.
\item
  Runtime has a weak positive correlation (0.049) with new vs old,
  meaning that older shows may have slightly longer runtimes.
\end{itemize}

Overall, the relationships between most variables are weak, indicating
little to no strong linear correlation between them.The only meaningful
correlation is between Number of episodes and episode quantityMany
(0.217), which makes sense as it reflects the classification of episode
quantity.

\subsection{4. Multicollinearity}\label{multicollinearity}

\begin{longtable}[]{@{}lr@{}}
\toprule\noalign{}
& VIF \\
\midrule\noalign{}
\endhead
\bottomrule\noalign{}
\endlastfoot
Number\_of\_episodes & 1.052561 \\
popularity & 1.005158 \\
runtime & 1.003248 \\
new\_vs\_old & 1.006478 \\
episode\_quantity & 1.054251 \\
\end{longtable}

\subsubsection{Multicollinearity
Analysis:}\label{multicollinearity-analysis}

All VIF values are close to 1, indicating that there is no significant
multicollinearity among the variables. This suggests that each variable
provides unique information to the model, and none of them are overly
redundant.

\end{document}
